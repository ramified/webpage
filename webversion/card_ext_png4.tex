
\documentclass{amsart}
%Typical documenttypes: article/book
%some examples:
%\documentclass[reqno,11pt]{book}   %%%for books
%\documentclass[]{minimal}			%%%for Minimal Working Example


%for beamers, you have to change a lot. Especially, remove the package enumitem!!!



%%%%%%%%%%%%%%%%%%%% setting for fast compiling

%\special{dvipdfmx:config z 0}		% no compression

\includeonly{chapters/chapter9}		% In practice, use an empty document called "chapter9"	% usually for printing books






%%%%%%%%%%%%%%%%%%%% here we include packages

%%%basic packages for math articles
\usepackage{amssymb}
\usepackage{amsthm}
\usepackage{amsmath}
\usepackage{amsfonts}
\usepackage[shortlabels]{enumitem}	% It supersedes both enumerate and mdwlist. The package option shortlabels is included to configure the labels like in enumerate.

%%%packages for special symbols
\usepackage{pifont}					% Access to PostScript standard Symbol and Dingbats fonts
\usepackage{wasysym}				% additional characters
\usepackage{bm}						% bold fonts: \bm{...}
\usepackage{extarrows}				% may be replaced by tikz-cd
%\usepackage{unicode-math}			% unicode maths for math fonts, now I don't know how to include it

%%%basic packages for fancy electronic documents
\usepackage[colorlinks]{hyperref}
\usepackage[table,hyperref]{xcolor} 			% before tikz-cd. 
%\usepackage[table,hyperref,monochrome]{xcolor}	% disable colored output (black and white)

%%%packages for figures and tables (general setting)
\usepackage{float}				%Improved interface for floating objects
\usepackage{caption,subcaption}
\usepackage{adjustbox}			% for me it is usually used in tables 
\usepackage{stackengine}		%baseline changes

%%%packages for commutative diagrams
\usepackage{tikz-cd}
%\usepackage{quiver}			% see https://q.uiver.app/
\usepackage{tikz}
\usetikzlibrary{calc,shapes}

%%%packages for pictures
\usepackage[width=0.5,tiewidth=0.7]{strands}
\usepackage{graphicx}			% Enhanced support for graphics

%%%packages for tables and general settings
\usepackage{array}
\usepackage{makecell}
\usepackage{multicol}
\usepackage{multirow}
\usepackage{diagbox}









%https://tex.stackexchange.com/questions/58852/possible-incompatibility-with-enumitem










%%%%%%%%%%%%%%%%%%%% here we include theoremstyles

\numberwithin{equation}{section}

\theoremstyle{plain}
\newtheorem{theorem}{Theorem}[section]

\newtheorem{setting}[theorem]{Setting}
\newtheorem{definition}[theorem]{Definition}
\newtheorem{lemma}[theorem]{Lemma}
\newtheorem{proposition}[theorem]{Proposition}
\newtheorem{corollary}[theorem]{Corollary}
\newtheorem{conjecture}[theorem]{Conjecture}

\newtheorem{claim}[theorem]{Claim}
\newtheorem{eg}[theorem]{Example}
\newtheorem{ex}[theorem]{Exercise}
\newtheorem{fact}[theorem]{Fact}
\newtheorem{ques}[theorem]{Question}
\newtheorem{warning}[theorem]{Warning}



\newtheorem*{bbox}{Black box}
\newtheorem*{notation}{Conventions and Notations}


\numberwithin{equation}{section}


\theoremstyle{remark}

\newtheorem{remark}[theorem]{Remark}
\newtheorem*{remarks}{Remarks}

%%% for important theorems
%\newtheoremstyle{theoremletter}{4mm}{1mm}{\itshape}{ }{\bfseries}{}{ }{}
%\theoremstyle{theoremletter}
%\newtheorem{theoremA}{Theorem}
%\renewcommand{\thetheoremA}{A}
%\newtheorem{theoremB}{Theorem}
%\renewcommand{\thetheoremB}{B}







%%%%%%%%%%%%%%%%%%%% here we declare some symbols

%%%%%%%DeclareMathOperator
%see here for why newcommand is better for DeclareMathOperator: https://tex.stackexchange.com/questions/67506/newcommand-vs-declaremathoperator

%%%%%basic symbols. Keep them!

%%%symbols for sets and maps
\DeclareMathOperator{\pt}{\operatorname{pt}}	%points. Other possibilities are \{pt\}, ...
\DeclareMathOperator{\Id}{\operatorname{Id}}	%identity in groups.
\DeclareMathOperator{\Img}{\operatorname{Im}}

\DeclareMathOperator{\Ob}{\operatorname{Ob}}
\DeclareMathOperator{\Mor}{\operatorname{Mor}}	%difference of Mor and Hom: Hom is usually for abelian categories
\DeclareMathOperator{\Hom}{\operatorname{Hom}}	\DeclareMathOperator{\End}{\operatorname{End}}
\DeclareMathOperator{\Aut}{\operatorname{Aut}}

%%%symbols for linear algebras and 
%%linear algebras
\DeclareMathOperator{\tr}{\operatorname{tr}}
\DeclareMathOperator{\diag}{\operatorname{diag}}	%for diagonal matrices

%%abstract algebras
\DeclareMathOperator{\ord}{\operatorname{ord}}
\DeclareMathOperator{\gr}{\operatorname{gr}}
\DeclareMathOperator{\Frac}{\operatorname{Frac}}

%%%symbols for basic geometries
\DeclareMathOperator{\vol}{\operatorname{vol}}	%volume
\DeclareMathOperator{\dist}{\operatorname{dist}}
\DeclareMathOperator{\supp}{\operatorname{supp}}

%%%symbols for category
%%names of categories
\DeclareMathOperator{\Mod}{\operatorname{Mod}}
\DeclareMathOperator{\Vect}{\operatorname{Vect}}
\DeclareMathOperator{\rep}{\operatorname{rep}} %usually rep means the category of finite dimensional representations, while Rep means the category of representations.
\DeclareMathOperator{\Rep}{\operatorname{Rep}}


%%%symbols for homological algebras
\DeclareMathOperator{\Tor}{\operatorname{Tor}}
\DeclareMathOperator{\Ext}{\operatorname{Ext}}
\DeclareMathOperator{\gldim}{\operatorname{gl.dim}}
\DeclareMathOperator{\projdim}{\operatorname{proj.dim}}
\DeclareMathOperator{\injdim}{\operatorname{inj.dim}}
\DeclareMathOperator{\rad}{\operatorname{rad}}


%%%symbols for algebraic groups
\DeclareMathOperator{\GL}{\operatorname{GL}}
\DeclareMathOperator{\SL}{\operatorname{SL}}

%%%symbols for typical varieties
\DeclareMathOperator{\Gr}{\operatorname{Gr}}
\DeclareMathOperator{\Flag}{\operatorname{Flag}}

%%%symbols for basic algebraic geometry
\DeclareMathOperator{\Spec}{\operatorname{Spec}}
\DeclareMathOperator{\Coh}{\operatorname{Coh}}
\newcommand{\Dcoh}{\mathcal{D}_{\operatorname{Coh}}}%%%This one shows the difference between \DeclareMathOperator and \newcommand
\DeclareMathOperator{\Pic}{\operatorname{Pic}}
\DeclareMathOperator{\Jac}{\operatorname{Jac}}

%%%%%advanced symbols. Choose the part you need!

%%%symbols for algebraic representation theory
\DeclareMathOperator{\Irr}{\operatorname{Irr}}
\DeclareMathOperator{\ind}{\operatorname{ind}}	%\ind(Q) means the set of  equivalence classes of finite dimensional indecomposable representations
\DeclareMathOperator{\Res}{\operatorname{Res}}
\DeclareMathOperator{\Ind}{\operatorname{Ind}}
\DeclareMathOperator{\cInd}{\operatorname{c-Ind}}


%%%symbols for algebraic topology
\DeclareMathOperator{\EGG}{\operatorname{E}\!}
\DeclareMathOperator{\BGG}{\operatorname{B}\!}

\DeclareMathOperator{\chern}{\operatorname{ch}^{*}}
\DeclareMathOperator{\Td}{\operatorname{Td}}
\DeclareMathOperator{\AS}{\operatorname{AS}}	%Atiyah--Segal completion theorem 

%%%symbols for Auslander--Reiten theory 
\DeclareMathOperator{\Modup}{\overline{\operatorname{mod}}}
\DeclareMathOperator{\Moddown}{\underline{\operatorname{mod}}}
\DeclareMathOperator{\Homup}{\overline{\operatorname{Hom}}}
\DeclareMathOperator{\Homdown}{\underline{\operatorname{Hom}}}


%%%symbols for operad
\DeclareMathOperator{\Com}{\operatorname{\mathcal{C}om}}
\DeclareMathOperator{\Ass}{\operatorname{\mathcal{A}ss}}
\DeclareMathOperator{\Lie}{\operatorname{\mathcal{L}ie}}
\DeclareMathOperator{\calEnd}{\operatorname{\mathcal{E}nd}} %cal=\mathcal


%%%%%personal symbols. Use at your own risk!

%%%symbols only for master thesis
\DeclareMathOperator{\ptt}{\operatorname{par}}	%the partition map
\DeclareMathOperator{\str}{\operatorname{str}}	%strict case
\DeclareMathOperator{\RRep}{\widetilde{\operatorname{Rep}}}
\DeclareMathOperator{\Rpt}{\operatorname{R}}
\DeclareMathOperator{\Rptc}{\operatorname{\mathcal{R}}}
\DeclareMathOperator{\Spt}{\operatorname{S}}
\DeclareMathOperator{\Sptc}{\operatorname{\mathcal{S}}}
\DeclareMathOperator{\Kcurl}{\operatorname{\mathcal{K}}}
\DeclareMathOperator{\Hcurl}{\operatorname{\mathcal{H}}}
\DeclareMathOperator{\eu}{\operatorname{eu}}
\DeclareMathOperator{\Eu}{\operatorname{Eu}}
\DeclareMathOperator{\dimv}{\operatorname{\underline{\mathbf{dim}}}}
\DeclareMathOperator{\St}{\mathcal{Z}}

%%%%%symbols which haven't been classified. Add your own math operators here!


\DeclareMathOperator{\Modr}{\operatorname{-Mod}}





%%%%%%%newcommand

%%%basic symbols
\newcommand{\norm}[1]{\Vert{#1}\Vert}

%%%symbols only for master thesis
\newcommand{\dimvec}[1]{\mathbf{#1}}
\newcommand{\abdimvec}[1]{|\dimvec{#1}|}
\newcommand{\ftdimvec}[1]{\underline{\dimvec{#1}}}

\newcommand{\absgp}[1]{\mathbb{#1}}
\newcommand{\WWd}{\absgp{W}_{\abdimvec{d}}}
\newcommand{\Wd}{W_{\dimvec{d}}}
\newcommand{\MinWd}{\operatorname{Min}(\absgp{W}_{\abdimvec{d}},W_{\dimvec{d}})}
\newcommand{\Compd}{\operatorname{Comp}_{\dimvec{d}}}
\newcommand{\Shuffled}{\operatorname{Shuffle}_{\dimvec{d}}}

\newcommand{\Omcell}{\Omega}
\newcommand{\OOmcell}{\boldsymbol{\Omega}}
\newcommand{\Vcell}{\mathcal{V}}
\newcommand{\VVcell}{\boldsymbol{\mathcal{V}}}
\newcommand{\Ocell}{\mathcal{O}}
\newcommand{\OOcell}{\boldsymbol{\mathcal{O}}}
\newcommand{\preimage}[1]{\widetilde{#1}}
\newcommand{\orde}{\operatorname{ord}_e}
\newcommand{\fakestar}{*}

%as the subscription of Hom
\newcommand{\Alggp}{\text{-Alg gp}}

%%%%%symbols which haven't been classified. Add your own math operators here!

\newcommand{\blockzero}[1]{\scalebox{1.3}{$\substack{\phantom{0}\\\phantom{0}#1\phantom{0}}$}}
\newcommand{\blockone}[1]{\scalebox{1.3}{${\substack{\phantom{0}\\#1\phantom{00}}}$}}
\newcommand{\blocktwo}[1]{\scalebox{1.3}{${\substack{\phantom{0}\\\phantom{00}#1}}$}}
\newcommand{\blockthree}[1]{\scalebox{1.3}{${\substack{#1\\\phantom{000}}}$}}
%%%%%%%%%%%%%%%%%%%% here we make some blocks for special features. 



%%%% todo notes %%%%
\usepackage[colorinlistoftodos,textsize=footnotesize]{todonotes}
\setlength{\marginparwidth}{2.5cm}
\newcommand{\leftnote}[1]{\reversemarginpar\marginnote{\footnotesize #1}}
\newcommand{\rightnote}[1]{\normalmarginpar\marginnote{\footnotesize #1}\reversemarginpar}









%%%%%%%%%%%%%%%%%%%% here we make some global settings. Understand everything here before you make a document!

\usepackage[a4paper,top=2cm, bottom=2cm, outer=0cm, inner=0cm]{geometry}
\usepackage{indentfirst}	% Indent first paragraph after section header

\setcounter{tocdepth}{2}


%https://latexref.xyz/_005cparindent-_0026-_005cparskip.html
\setlength{\parindent}{15pt}	
\setlength{\parskip}{0pt plus1pt}

%\setlength\intextsep{0cm}
%\setlength\textfloatsep{0cm}
\def\arraystretch{1}
%\setcounter{secnumdepth}{3}

\allowdisplaybreaks
\thispagestyle{empty}

\begin{document}

% The beginning depends on the documentclass. Rewrite this part if you use different documentclass!



%\tikz[remember picture,overlay] \node[opacity=0.5,inner sep=0pt] at (current page.center){\includegraphics[width=\paperwidth,height=\paperheight]{figures/background.png}};
\tikzset{
% Defines a custom style which generates BOTH, .pdf and .png export
% but prefers the .png on inclusion.
%%This style is not pre-defined, you may need to copy-paste and
% adjust it.
png export/.style={
external/system call/.add=
{}
{& magick.exe -density 300 -transparent white "\image.pdf" "\image.png"},
%
/pgf/images/external info,
/pgf/images/include external/.code={%
\includegraphics
[width=\pgfexternalwidth,height=\pgfexternalheight]
{##1.png}%
},
}
}
%\tikzset{
%png export/.style={
%    % First we call ImageMagick; change settings to requirements
%    external/system call/.add={}{& magick.exe -density 300 "\image.pdf" "\image.png"},
%    % Now we force the PNG figure to be used instead of the PDF
%    /pgf/images/external info,
%    /pgf/images/include external/.code={
%        \includegraphics[width=\pgfexternalwidth,height=\pgfexternalheight]{##1.png}
%    },
%}
%}
{
% Here we specify the figure will be converted and inserted as PNG
\tikzset{png export}
\begin{tikzpicture} [remember picture, overlay]
\def\border{1.5cm}
\def\upborder{0.7cm}
\def\height{4.5cm}
\def\betweenpieces{13mm}
\def\extraheight{15mm}
\def\cornerradius{2mm}
\def\distance{0}
\def\borderlinewidth{0.3mm}
\def\borderlinewhitewidth{0.8mm}
\def\borderlinewholewidth{\borderlinewidth *2 + \borderlinewhitewidth}
\foreach \x/\xcolor in {3/green}
{
\node (PosA\x) at ([xshift=\border,yshift=-(\upborder+\x * \distance)]current page.north west){};
\node (PosC\x) at ([xshift=\border,yshift=-(\upborder+\height+\x *\distance)]current page.north west){};
\node (PosD\x) at ([xshift=-\border,yshift=-(\upborder+\height+\x *\distance)]current page.north east){};
\draw (PosA\x) [line width=\borderlinewholewidth,black,rounded corners=\cornerradius]rectangle (PosD\x) node(recCen\x)[midway] {};

\draw[line width=\borderlinewholewidth,rounded corners=\cornerradius]
(PosC\x) ++(36mm+ 2 * \x * \betweenpieces ,0) --++(\cornerradius,0)-- ++(0,-\extraheight) --++(2 * \betweenpieces,0) -- ++(0,\extraheight) --++(\cornerradius,0);

\draw (PosA\x) [line width=\borderlinewhitewidth,\xcolor,rounded corners=\cornerradius]rectangle (PosD\x);
\draw[line width=\borderlinewhitewidth,\xcolor,rounded corners=\cornerradius]
    (PosC\x) ++(36mm+ 2 * \x * \betweenpieces ,0) --++(\cornerradius,0)-- ++(0,-\extraheight) --++(\betweenpieces,0)node(PosE\x)[yshift=\extraheight/2,black] {}--++(\betweenpieces,0) -- ++(0,\extraheight) --++(\cornerradius,0);

\draw (PosE\x) [yshift=-3mm,fill=black]circle (0.6mm)node(PosF0\x){};
\draw (PosF0\x) [xshift=-7mm,fill=black]circle (0.6mm)node(PosF1\x){};
\draw (PosF0\x) [xshift=7mm,fill=black]circle (0.6mm)node(PosF2\x){};
\draw (PosF0\x) [yshift=7mm,fill=black]circle (0.6mm)node(PosF3\x){};
\draw (PosF\x\x)[thick] circle (1.2mm);
\draw (PosF0\x) -- (PosF1\x);
\draw (PosF0\x) -- (PosF2\x);
\draw (PosF0\x) -- (PosF3\x);
}
%\coordinate [label=left:$C$] (PosC0) at ([yshift=-\height]PosA0);
% https://q.uiver.app/?q=WzAsMjcsWzIsMSwiXFxzdWJzdGFja3swXFxcXDAwMH0iXSxbMywxLCJcXHN1YnN0YWNrezBcXFxcMDAwfSJdLFsyLDAsIlxcc3Vic3RhY2t7MFxcXFwwMDB9Il0sWzIsMiwiXFxzdWJzdGFja3swXFxcXDAwMH0iXSxbMSwxLCJcXHN1YnN0YWNrezBcXFxcMDAwfSJdLFswLDAsIlxcc3Vic3RhY2t7MFxcXFwwMDB9Il0sWzAsMSwiXFxzdWJzdGFja3swXFxcXDAwMH0iXSxbMCwyLCJcXHN1YnN0YWNrezBcXFxcMDAwfSJdLFs0LDIsIlxcc3Vic3RhY2t7MFxcXFwwMDB9Il0sWzQsMSwiXFxzdWJzdGFja3swXFxcXDAwMH0iXSxbNCwwLCJcXHN1YnN0YWNrezBcXFxcMDAwfSJdLFs1LDEsIlxcc3Vic3RhY2t7MFxcXFwwMDB9Il0sWzYsMiwiXFxzdWJzdGFja3swXFxcXDAwMH0iXSxbNiwxLCJcXHN1YnN0YWNrezBcXFxcMDAwfSJdLFs2LDAsIlxcc3Vic3RhY2t7MFxcXFwwMDB9Il0sWzcsMSwiXFxzdWJzdGFja3swXFxcXDAwMH0iXSxbOCwwLCJcXHN1YnN0YWNrezBcXFxcMDAwfSJdLFs4LDEsIlxcc3Vic3RhY2t7MFxcXFwwMDB9Il0sWzgsMiwiXFxzdWJzdGFja3swXFxcXDAwMH0iXSxbOSwxLCJcXHN1YnN0YWNrezBcXFxcMDAwfSJdLFsxMCwwLCJcXHN1YnN0YWNrezBcXFxcMDAwfSJdLFsxMCwxLCJcXHN1YnN0YWNrezBcXFxcMDAwfSJdLFsxMCwyLCJcXHN1YnN0YWNrezBcXFxcMDAwfSJdLFsxMSwxLCJcXHN1YnN0YWNrezBcXFxcMDAwfSJdLFsxMiwwLCJcXHN1YnN0YWNrezBcXFxcMDAwfSJdLFsxMiwxLCJcXHN1YnN0YWNrezBcXFxcMDAwfSJdLFsxMiwyLCJcXHN1YnN0YWNrezBcXFxcMDAwfSJdLFswLDFdLFs1LDRdLFs2LDRdLFs3LDRdLFs0LDJdLFs0LDBdLFs0LDNdLFsyLDFdLFsxLDEwXSxbMSw5XSxbMywxXSxbMSw4XSxbMTAsMTFdLFs5LDExXSxbOCwxMV0sWzExLDE0XSxbMTEsMTNdLFsxMSwxMl0sWzE0LDE1XSxbMTMsMTVdLFsxMiwxNV0sWzE1LDE2XSxbMTUsMTddLFsxNSwxOF0sWzE2LDE5XSxbMTcsMTldLFsxOCwxOV0sWzE5LDIwXSxbMTksMjFdLFsxOSwyMl0sWzIyLDIzXSxbMjEsMjNdLFsyMCwyM10sWzIzLDI0XSxbMjMsMjVdLFsyMywyNl1d



    \node [black] at (recCen3) {
$\begin{tikzcd}[sep={between origins, \betweenpieces}]
      	\blockthree{0} && \blockthree{0} && \blockthree{0} && \blockthree{1} && \blockthree{0} && \blockthree{0} && \blockthree{0} \\
      	\blockthree{0} & \blockthree{0} & \blockthree{0} & \blockthree{0} & \blockthree{1} & \blockthree{1} & \blockthree{0} & \blockthree{1} & \blockthree{1} & \blockthree{0} & \blockthree{0} & \blockthree{0} & \blockthree{0} \\
      	\blockthree{0} && \blockthree{0} && \blockthree{0} && \blockthree{1} && \blockthree{0} && \blockthree{0} && \blockthree{0}
      	\arrow[from=2-3, to=2-4]
      	\arrow[from=1-1, to=2-2]
      	\arrow[from=2-1, to=2-2]
      	\arrow[from=3-1, to=2-2]
      	\arrow[from=2-2, to=1-3]
      	\arrow[from=2-2, to=2-3]
      	\arrow[from=2-2, to=3-3]
      	\arrow[from=1-3, to=2-4]
      	\arrow[from=2-4, to=1-5]
      	\arrow[from=2-4, to=2-5]
      	\arrow[from=3-3, to=2-4]
      	\arrow[from=2-4, to=3-5]
      	\arrow[from=1-5, to=2-6]
      	\arrow[from=2-5, to=2-6]
      	\arrow[from=3-5, to=2-6]
      	\arrow[from=2-6, to=1-7]
      	\arrow[from=2-6, to=2-7]
      	\arrow[from=2-6, to=3-7]
      	\arrow[from=1-7, to=2-8]
      	\arrow[from=2-7, to=2-8]
      	\arrow[from=3-7, to=2-8]
      	\arrow[from=2-8, to=1-9]
      	\arrow[from=2-8, to=2-9]
      	\arrow[from=2-8, to=3-9]
      	\arrow[from=1-9, to=2-10]
      	\arrow[from=2-9, to=2-10]
      	\arrow[from=3-9, to=2-10]
      	\arrow[from=2-10, to=1-11]
      	\arrow[from=2-10, to=2-11]
      	\arrow[from=2-10, to=3-11]
      	\arrow[from=3-11, to=2-12]
      	\arrow[from=2-11, to=2-12]
      	\arrow[from=1-11, to=2-12]
      	\arrow[from=2-12, to=1-13]
      	\arrow[from=2-12, to=2-13]
      	\arrow[from=2-12, to=3-13]
\end{tikzcd}$
};    

\end{tikzpicture}
}
\include{chapters/chapter9}


\end{document}